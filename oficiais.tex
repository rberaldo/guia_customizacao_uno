\section{Regras Oficiais}

A seguir serão apresentadas as regras oficiais do jogo. Elas serão fielmente reproduzidas do folheto original que acompanhas as embalagens brasileiras do jogo, distribuído pela empresa Mattel. 

\subsection{Regras em Verbatim}

\vspace{0.5cm}

\textsc{\large{Conteúdo}}

\textbf{108 cartas, sendo}

\textbf{19 Cartas azuis} --- 0 a 9

\textbf{19 Cartas Verdes} --- 0 a 9

\textbf{19 Cartas Vermelhas} --- 0 a 9

\textbf{19 Cartas Amarelas} --- 0 a 9

\textbf{8 Cartas Comprar Duas Cartas} --- 2 de cada em azul, verde, vermelho e amarelo

\textbf{8 Cartas Inverter} --- 2 de cada em azul, verde, vermelho e amarelo

\textbf{8 Cartas Pular} --- 2 de cada em azul, verde, vermelho e amarelo

\textbf{4 Curingas}

\textbf{4 Curingas Comprar Quatro Cartas}

\vspace{0.5cm}

\textsc{\large{Objetivo do Jogo}}

Ser o primeiro jogador a marcar 500 pontos. Os pontos são marcados livrando-se de todas as cartas que estão na sua mão antes do adversário. Você marca pontos pelas cartas que ficaram nas mãos dos seus adversários.

\vspace{0.5cm}

\textsc{\large{Como Jogar}}

Cada jogador compra uma carta. A pessoa que tirar a carta mais alta faz a distribuição. Cartas de ação contam como zero nesta parte do jogo.

Quando as cartas estiverem embaralhadas, são distribuídas 7 cartas para cada jogador.

As cartas restantes são colocadas viradas para baixo para formar o MONTE. A carta de cima do MONTE é virada para cima para começar uma pilha de DESCARTE. Se uma Carta de Ação for a primeira a ser aberta no MONTE, são adotadas regras específicas (veja FUNÇÕES DAS CARTAS DE AÇÃO).

O jogador à esquerda do distribuidor começa o jogo. Ele precisa jogar uma carta que combine com a que está na pilha de DESCARTE em valor, cor ou naipe. Por exemplo, se a carta for um \textit{7 vermelho}, o jogador precisa descartar uma carta \textit{vermelha} ou um \textit{7 de qualquer cor}. Alternativamente, o jogador pode descartar um \textit{Curinga} (Veja FUNÇÕES DAS CARTAS DE AÇÃO).

Se o jogador não tiver uma carta que combine com a que está na pilha de DESCARTE, ele \textit{precisa} comprar uma carta do MONTE. Se a carta comprada servir, o jogador pode baixá-la na mesma vez. Caso contrário, a vez passa para o próximo jogador.

Os jogadores podem optar por não jogar uma carta válida que esteja na mão dele. Nesse caso, o jogador precisa comprar uma carta do MONTE. Se esta puder ser jogada, a carta pode ser descartada na mesma vez, mas o jogador não pode jogar uma carta que estiver na sua mão depois de ter feito a compra.

\vspace{1.25cm}

\textsc{\large{Funções das Cartas de Ação}}

As funções das Cartas de Ação, e quando podem ser jogadas, estão definidas abaixo.

\textbf{Carta Comprar Duas Cartas} --- Quando esta carta é jogada, o próximo jogador \textit{precisa comprar 2 cartas} e \textit{perderá a vez}. Esta carta só pode ser jogada sobre uma de mesma cor e sobre outras cartas Comprar Duas Cartas. Se ela for aberta no começo do jogo, as mesmas regras se aplicam.

\textbf{Carta Inverter} --- Ela simplesmente inverte a direção do jogo. Se o jogo estiver girando para a esquerda, passa a girar para a direita, e vice-versa. A carta só pode ser jogada sobre uma da mesma cor ou sobre outra carta Inverter. Se esta carta for aberta no começo do jogo, o distribuidor começa e, depois, o jogo se desloca para a direita em vez de se deslocar para a esquerda.

\textbf{Carta Pular} --- O jogador seguinte àquele que descartou essa carta perde a vez --- é ``\textit{pulado}''. A carta só pode ser jogada sobre uma carta da mesma cor ou sobre outra carta Pular. Se uma carta Pular for aberta no começo do jogo, o jogador à esquerda do distribuidor é ``pulado'', e o jogador à esquerda deste último começa o jogo.

\textbf{Curinga} --- A pessoa que jogar esta carta exige \textit{qualquer} cor para continuar a jogar, inclusive a cor que está sendo jogada no momento, se assim desejar. Um Curinga pode ser jogado a qualquer hora --- mesmo que o jogador tenha na mão outra carta que possa ser jogada. Se um Curinga for aberto no começo do jogo, o jogador à esquerda do distribuidor determina a cor que dará sequência ao jogo.

\textbf{Curinga Comprar Quatro Cartas} --- Esta é a melhor carta. A pessoa que jogar esta carta exige a cor que continua o jogo. Além disso, o próximo jogador precisa \textit{comprar 4 cartas} do MONTE e \textit{perde a vez}. Infelizmente, esta carta só pode ser jogada quando o jogador que a possui não tiver na mão uma carta de mesma cor da que está na pilha de DESCARTE. Entretanto, se o jogador tiver cartas com o mesmo número ou Cartas de Ação, o Curinga Comprar Quatro Cartas pode ser jogado. Um jogador que possuir um Curinga Comprar Quatro Cartas pode tentar blefar e jogar uma carta inapropriada, mas certas regras se aplicam se ele for apanhado (veja PENALIDADES). Se esta carta for aberta no começo do jogo, é devolvida ao baralho e uma outra carta é aberta.

\vspace{0.5cm}

\textsc{\large{Saindo}}

Quando só restar uma carta ao jogador, ele precisa gritar ``UNO'' (que significa ``uma''). Se ele não o fizer, terá de comprar 2 cartas do MONTE. Entretanto, isto só é necessário se ele for apanhado por um dos outros jogadores (veja PENALIDADES).

Quando o jogador não tiver mais nenhuma carta, a partida acabou. São marcados os pontos (veja PONTUAÇÃO) e o jogo recomeça.

Se a última carta jogada em uma partida for a carta Comprar Duas Cartas ou o Curinga Comprar Quatro Cartas, o próximo jogador \textit{precisa} comprar 2 ou 4 cartas, respectivamente. Estas cartas são contadas na totalização dos pontos.

Se nenhum jogador estiver sem cartas quando o MONTE acabar, o baralho é novamente embaralhado e o jogo continua.

\vspace{0.5cm}

\textsc{\large{Pontuação}}

O primeiro jogador que se livrar das cartas recebe pontos pelas cartas que \textit{ficaram} nas mãos dos adversários, como segue:

Todas as cartas numeradas (0--9) --- Valor Nominal

Comprar Duas Cartas --- 20 Pontos

Inverter --- 20 Pontos

Pular --- 20 Pontos

Curinga --- 50 Pontos

Curinga Comprar Quatro Cartas --- 50 Pontos

\vspace{0.5cm}

\textsc{\large{Vencendo o Jogo}}

O VENCEDOR é o primeiro jogador que fizer 500 pontos. Entretanto, os pontos do jogo podem ser marcados mantendo um total parcial dos pontos com que cada jogador é apanhado no fim de cada partida. Quando um jogador alcançar os 500 pontos, o jogador com menos pontos é o vencedor, ou seja, aquele que não deixar sobrarem muitas cartas na mão.

\vspace{0.5cm}

\textsc{\large{Penalidades}}

O jogador que esquecer de dizer ``UNO'' antes que a penúltima carta toque a pilha de DESCARTE, mas se lembrar (e gritar ``UNO'') antes que qualquer outro jogador ``perceba'', está salvo e não está sujeito à penalidade. Os jogadores não podem ser apanhados por não dizer ``UNO'' depois que o próximo jogador começar a sua vez. ``Começar a vez'' é definido como comprar uma carta do MONTE ou tirar uma carta da mão para jogar.

Jogadores que derem palpites aos outros jogadores \textit{precisam} comprar 2 cartas do MONTE.

Se um jogador jogar uma carta errada e isto for notado por qualquer um dos demais jogadores, ele ``\textit{precisa}'' pegá-la de volta e comprar mais 2 cartas extras do MONTE. O jogo continua com o próximo jogador.

Se um Curinga Comprar Quatro Cartas for exibido indevidamente (isto é, se o jogador possuir uma carta da mesma cor da que está na pilha de DESCARTE) e a pessoa que estiver jogando for desafiada, a mão \textit{precisa} ser mostrada primeiro ao jogador que fez o desafio.

Se o Curinga Comprar Quatro Cartas tiver sido jogado indevidamente, o jogador responsável precisa comprar 4 cartas. Se a carta foi jogada corretamente, o desafiante precisa comprar 2 cartas, além das 4. O desafio só pode ser feito pelo jogador obrigado a comprar 4 cartas depois que o Curinga Comprar Quatro Cartas tiver sido descartado na mesa.

\vspace{0.5cm}

\textsc{\large{Jogo em Dupla, Parceiros e Torneios em Várias Mesas}}

\textbf{Jogo em Dupla (jogo UNO com dois jogadores)}

Joga-se com as seguintes regras especiais:

\begin{enumerate}
\item{Jogar uma carta Inverter funciona como Pular. O jogador que joga a carta Inverter pode imediatamente jogar outra carta.}
\item{A pessoa que está jogando uma carta Pular pode imediatamente jogar outra carta.}
\item{Quando uma pessoa joga uma carta Comprar Duas Cartas e o outro jogador tiver comprado as 2 cartas, o jogo volta ao primeiro jogador. O mesmo princípio se aplica à carta Curinga Comprar Quatro Cartas. As regras comuns do jogo de cartas UNO se aplicam aos demais casos.}
\end{enumerate}

\textbf{Parceiros}

Os parceiros sentam-se de frente um para o outro. Quando um dos parceiros ficar sem cartas, a partida terminou. Todos os pontos nas mãos dos dois parceiros adversários são somados e passados para o time vencedor.

\textbf{Variações}

Com quatro jogadores podem ser jogadas quatro partidas e em cada uma mudam os parceiros. Todos os jogadores mantêm-se a par dos pontos feitos em cada uma das suas parcerias. Podem ser jogadas diversas rodadas e a pessoa que fizer mais pontos é declarada a vencedora.

Com oito jogadores, podem ser jogados dois jogos separados em duas mesas e cada jogador tem outro jogador como parceiro durante quatro partidas (um total de 28 partidas). Pontuação como acima.

\textbf{Jogo de Cartas DESAFIO UNO}

Este jogo é pontuado mantendo-se um total parcial das cartas que estão na mão de cada jogador. Quando um jogador alcança um valor combinado, possivelmente 500, o jogador é eliminado do jogo. Quando restarem apenas dois jogadores, eles jogam de igual pra igual. O jogador que alcançar ou superar o valor combinado perde. O ganhador da última partida é declarado o vencedor do jogo (veja regras especiais para JOGO EM DUPLA).

\textbf{Divirta-se com UNO --- e que vença o melhor ou aquele com mais sorte!}