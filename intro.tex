\section*{Introdução}
\addcontentsline{toc}{section}{Introdução}

Este guia é destinado a todas aquelas pessoas que já se encontraram em situações difíceis por ocasião das regras deste jogo incrível e flexível que é UNO.

À bem da verdade, o UNO foi uma encarnação nova para um jogo antigo: a ideia é muito óbvia e simples para acharmos que ele é um jogo original. Entretanto, ao invés de usarmos as cartas de figura do baralho normal (Rei, Rainha, Valete) para representar as ações como "pular o próximo jogador", "comprar duas cartas", etc, o baralho de UNO foi criado especificamente para este tipo de jogo. Esse é um apelo visual que torna o jogo ainda mais coerente e atraente, e não apenas algo inventado, usando-se de gambiarras com o velho e cinzento baralho normal.

Esta invenção, contudo, não deve ser vista como muito recente; ela existe desde a década de 70 nos Estados Unidos, e atualmente existem dezenas de variações de UNO --- versões em que você pode jogar na piscina, ou usando uma arma, mais ou menos como um Paintball, mas com cartas. Não tratamos de nenhuma dessas aqui: nos concentramos no bom e velho jogo de cartas e como suas regras podem ser flexibilizadas.

Apesar de não ser um jogo novo e revolucionário, UNO é divertido, viciante, \textit{encantador}. Sua simplicidade é um duplo fardo: mesmo ao comprarmos um baralho novo em folha, passamos os olhos pela folha de regras que o acompanha, damos uma olhada em como são as regras em francês, e então jogamos a folha no lixo ou em alguma gaveta escura. Isso faz com que muitas regras sejam distorcidas, e uma miríade de novas regras entrem em campo, fazendo com que quase nunca se jogue o mesmo UNO em grupos de jogadores diferentes.

Entretanto, isso de forma alguma é um problema; o jogo não se modifica tanto a ponto de não ser ``o velho UNO''. É por isso que esse guia foi escrito: para ajudar a fazer uma escolha consciente quanto às diferentes regras que podem ser aplicadas para tornar o jogo mais agradável de acordo com as preferências dos jogadores.

Ele possui algumas seções distintas: em primeiro lugar teremos contato com as regras oficiais do jogo. Como ele deve ser jogado. Logo depois veremos como a interpretação pode entrar em cena, já bifurcando levemente tipos distintos de jogo.

Entraremos então em contato com a essência estratégica do jogo. O que se deve fazer para ganhar? Quais as melhores artimanhas, qual a lógica a ser seguida?

Então poderemos discutir as regras alternativas, cada uma em sua seção: regras que mudam o modo de jogar ou adicionam funcionalidades às cartas, ou mesmo mudam as condições de vitória do jogo. Cada uma é discutida em suas consequências para a dinâmica e estratégia do jogo.

Bom divertimento!