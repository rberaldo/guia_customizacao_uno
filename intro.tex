\chapter*{Introdução}
\addcontentsline{toc}{chapter}{Introdução}

Este guia é destinado a todas aquelas pessoas que já se encontraram em situações difíceis por ocasião das regras deste jogo incrível e flexível que é UNO.

À bem da verdade, o UNO foi uma encarnação nova para um tipo de jogo antigo. Ao invés de usarmos as cartas de figura do baralho normal (Rei, Rainha, Valete) para representar as ações como "pular o próximo jogador", "comprar duas cartas", etc, o baralho de UNO foi criado especificamente para este tipo de jogo. Esse é um apelo visual que torna o jogo ainda mais coerente e atraente, e não apenas algo inventado, usando-se de gambiarras com o velho e cinzento baralho normal. Esta encarnação existe desde a década de 70 nos Estados Unidos, e atualmente existem dezenas de variações de UNO --- versões em que você pode jogar na piscina, ou usando uma aparelho que ``cospe'' as cartas. Não tratamos de nenhuma dessas aqui: nos concentramos no bom e velho jogo de cartas e como suas regras podem ser modificadas.

Apesar de não ser um jogo novo e revolucionário, UNO é divertido, viciante, encantador. Sua simplicidade é um duplo fardo: mesmo ao comprarmos um baralho novo em folha, passamos os olhos pelo papel com regras que o acompanha, damos uma olhada em como são as regras em francês, apenas por curiosidade, e então jogamos a folha no lixo ou em alguma gaveta escura. Isso faz com que muitas regras sejam distorcidas, e uma miríade de novas regras entrem em campo, fazendo com que quase nunca se jogue o mesmo UNO em grupos de jogadores diferentes.

Entretanto, isso de forma alguma é um problema; o jogo não se modifica tanto a ponto de não ser ``o velho UNO''. É por isso que esse guia foi escrito: para ajudar a fazer uma escolha consciente quanto às diferentes regras que podem ser aplicadas para tornar o jogo mais agradável de acordo com as preferências (e objetivos) dos jogadores.

Ele possui algumas seções distintas: em primeiro lugar teremos contato com as regras oficiais do jogo. Logo depois veremos como a interpretação pode entrar em cena, já bifurcando levemente tipos distintos de UNO. Entraremos então em contato com a essência estratégica do jogo. O que se deve fazer para ganhar? Quais são as melhores artimanhas? Então poderemos enfim discutir as regras alternativas, cada uma em sua seção: regras que mudam o modo de jogar ou adicionam funcionalidades às cartas, ou mesmo mudam as condições de vitória do jogo. Cada uma é discutida em suas consequências para a dinâmica e estratégia do jogo, variações e conexões que fazem com outras regras alternativas.

Se você está lendo este Guia através do PDF, existem alguns truques para facilitar sua vida: no Sumário, cada ítem é um link; clique ele e será levado diretamente ao início da seção escolhida. Quando alguma referência a outra parte do guia for feita, o número de página também será um link evidenciado por um retângulo vermelho: clique nele e você será levado à página referenciada.

Se você gostou de alguma regra alternativa, decore seu código para referência posterior. Assim, apenas pesquise por [CÓDIGO] (entre colchetes e com as letras do código maísculas) para encontrar a descrição detalhada da regra, ou por (CÓDIGO) (entre parênteses, dessa vez) para encontrar uma rápida explicação no apêndice que resume as regras alternativas.

Bom divertimento!