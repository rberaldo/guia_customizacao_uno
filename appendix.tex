\chapter{Apêndices}

\appendix

\section{Resumo das Estratégias do Jogo Oficial 1121}

\begin{itemize}
\item{O jogo é um constante equilíbrio pessoal entre jogar pela continuidade (para livrar-se das cartas na mão e vencer o mais rapidamente possível) e jogar pelo gerenciamento (para controlar as jogadas dos outros e evitar a vitória deles).}
\item{Quanto menor o número de cartas, menores as possibilidades de gerenciar o jogo, e vice-versa.}
\item{Guardar cartas importantes para um outro momento do jogo é uma boa estratégia, mas guardá-las por tempo demais é perigoso: Curingas contam 50 pontos no jogo e cartas de ação, 20.}
\item{Ao jogar de acordo com o princípio da continuidade, jogue sempre para tentar manter o maior número de possibilidades para a próxima rodada na sua mão. Dessa forma, se você pode escolher entre jogar uma carta A ou a carta B, jogue de forma a preservar suas possibilidades; se ao jogar a carta A você acaba com todas as cartas de uma determinada cor na sua mão, e isso não ocorre com a carta B, jogue a carta B.}
\item{Se com nenhuma delas você prejudica a variedade de cores na sua mão, você precisará ter uma boa noção do que já foi jogado antes para forjar uma estratégia mais consciente: jogue uma cor que já tenha sido bastante jogada (menos chance de que ela volte na próxima rodada) ou jogue um número já bastante jogado (menos chance de que eventualmente ele apareça de novo). Prefira sempre acabar com a variedade de cartas de mesmo número na mão do que com a variedade de cores.}
\end{itemize}

\newpage

\section{Arranjo das Cartas na Mão}

Aqui vai um conselho pessoal quanto à ordenação de cartas na mão:

\begin{itemize}
\item{Em primeiro lugar, cuidado com a postura, a posição da mão e com superfícies espelhadas: não mostre suas cartas aos oponentes!}
\item{Depois, ordene as cartas numéricas em ordem crescente da esquerda para a direita. Se houver mais de uma carta do mesmo número, procure deixar as cartas de um jeito que de uma carta para outra fique a mesma cor. Por exemplo, se você tiver um 1 Amarelo, um 1 Verde, um 3 Amarelo, um 3 Azul e um 4 Azul, deixe as cartas na ordem 1 Verde, 1 Amarelo, 3 Amarelo (para que duas cartas amarelas fiquem juntas), 3 Azul, 4 Azul (aqui temos as cartas azuis juntas também).}
\item{Insira as cartas de ação coloridas próximo a um grande grupo de cartas com a mesma cor (se possível), mas nunca entre duas cartas de mesmo número. Ou, se preferir, insira todas no início ou no final da mão.}
\item{Insira os Curingas entre duas cartas de valores numéricos diferentes e cores diferentes. Ou, se preferir, insira todas no início ou no final da mão.}
\end{itemize}

\newpage

\section{Resumo das Regras Alternativas}

\begin{center}
\begin{longtable}{|p{4cm}|p{5cm}|}
\hline
Torneio de Vitórias (TV) & Conta-se o número de vitórias, não os pontos. \\ \hline
Pontos Reversos (PR) & Todos começam com 500 e vão perdendo pontos. \\ \hline
Morte Súbita (MS) \emph{e variações} & Quem precisar comprar uma carta está eliminado da partida. \\ \hline
Final Limpo (FL) \emph{e variações} & É proibido vencer o jogo com Curingas ou cartas que forcem outros jogadores a comprar cartas. \\ \hline
Flash UNO (FU) \emph{e variações} & Todos começam o jogo com 4 cartas. \\ \hline
Battlefield UNO (BU) \emph{e variações} & Todos começam o jogo com 9 cartas. \\ \hline
Corte (C) & Uma carta idêntica à da pilha de descarte pode ser jogada por qualquer jogador, na vez de jogar ou não. O jogo prossegue a partir do jogador que fez o corte. \\ \hline
Descarte Rápido (DR) & Duas cartas de número ou símbolo iguais podem ser descartadas na mesma vez. \\ \hline
Duplicatas (DD) & Duas cartas idênticas podem ser descartadas na mesma vez. \\ \hline
Primeira Regra da Aritmética (AA) & Duas cartas numéricas de mesma cor podem ser descartadas se sua soma ou subtração (mais que duas cartas no caso da soma) resultarem no número da carta (necessariamente numérica) que está na pilha de descarte. \\ \hline
Segunda Regra da Aritmética (AB) & Uma carta numérica pode ser jogada se for a soma das duas últimas cartas na pilha de descarte se estas forem de mesma cor. \\ \hline
Satisfação (SAT) & É obrigatório comprar até ser comprada uma carta que possa ser jogada. \\ \hline
Consolo (CON) & Cartas compradas por causa de cartas Comprar Duas ou Comprar Quatro podem ser jogadas na mesma vez. \\ \hline
Insatisfação (INS) & Uma carta comprada nunca pode ser jogada. \\ \hline
Duplo Descarte (DED) & Os jogadores podem escolher descartar em uma pilha de descarte em que as cartas são visíveis para todos os jogadores, ou ocultar sua jogada em uma pilha de descarte em que as cartas são viradas para baixo. Punições são aplicadas se os jogadores descobrirem que uma jogada não-legítima foi feita. \\ \hline
Gêmeo Traidor (GT) & Se uma carta numérica idêntica à que está na pilha de descarte for jogada, o jogador que jogou a primeira carta compra o número de cartas da carta jogada. \\ \hline
Passe (P) \emph{e variações} & Cartas Comprar Duas Cartas e Curinga Comprar Quatro Cartas podem ser jogadas uma em cima da outra, acumulando. Variações decidem quais cartas podem ou não ser acumuladas. \\ \hline
Velocidade (V) \emph{e variações} & Todas as jogadas precisam ser feitas em 5 segundos. \\ \hline
Primeira Regra da Paz (PAA) \emph{e variações} & Ninguém pode descartar uma carta de ação nas primeiras rodadas do jogo. \\ \hline
Inferno (I) & Qualquer carta de ação pode ser jogada em cima de qualquer carta de ação. \\ \hline
Regra de Chukcki (CHU) & Se duas cartas de mesmo valor numérico forem jogadas consecutivamente por jogadores distintos, o terceiro jogador deve mudar a cor na pilha de descarte. \\ \hline
Batismo (BA) & Depois de duas cartas de valor numérico igual jogadas consecutivamente por jogadores diferentes, se o terceiro jogador jogar uma carta Inverter todos compram 1 carta. \\ \hline
Revelação (REV) & O MONTE fica virado para cima. \\ \hline
Silêncio (S) \emph{e variações} & A cada 7 jogado, os jogadores não podem falar até que outro 7 seja jogado. \\ \hline
Carrossel (CAR) & A cada 0 jogado, os jogadores trocam de mão, de forma que cada jogador passa sua mão para o próximo jogador na sequência. \\ \hline
Troca (T) & A cada 0 jogado, o jogador que o jogou pode escolher trocar de mão com outro. \\ \hline
Monte (M) & Quando um 8 é jogado os jogadores devem pôr a mão sobre a pilha de descarte. \\ \hline
Elogios (E) & Um elogio por adversário é feito a quem jogar um 5 Verde. \\ \hline
Regra Invertida dos Elogios (IE) & Uma crítica por adversário é feita a quem jogar um 5 Vermelho. \\ \hline
Doações (DOA) & O jogador que descartar uma carta 2 escolhe um adversário para o qual doar duas de suas cartas. \\ \hline
Segunda Regra da Paz (PAB) & Se uma carta 1 é jogada, não se pode jogar cartas de ação até que outro 1 seja jogado. \\ \hline 
Regra do Dado (DAD) & Se uma carta 4 é jogada, o próximo jogador deve jogar um dado e comprar o número de cartas equivalente ao resultado do dado. \\ \hline
Regra da Trindade (TRI) & Se uma carta 3 é jogada, o próximo jogador deve jogar uma outra carta 3 ou uma carta de mesma cor, sob pena de comprar 3 cartas. \\ \hline 
Regra do Backlash (BACK) & Se uma carta 8 é jogada, o jogador anterior deve comprar uma carta. \\ \hline 
Regra da Clarividência (CLA) & O jogador que joga uma carta 1 escolhe um jogador que deve mostrar a ele suas cartas. \\ \hline 
Movimento (MOV) & A cada fim de partida os jogadores devem mudar de ordem. \\ \hline
Desafio Polivalente (DP) & Os jogadores podem desconfiar quando um jogador compra uma carta. Existem penalidades para o caso de não haver necessidade de fazê-lo. \\ \hline
Regra de Dori (MEM) & Quem esquecer de dizer UNO deve comprar quatro cartas. \\ \hline
Regra da Prescrição (PRE) & Uma pessoa pode ser acusada de não ter dito UNO a qualquer momento após ter ficado UNO. \\ \hline
Regra do Orgulho (ORG) & Comprará duas cartas aquele que pedir desculpas durante a partida. \\ \hline
\end{longtable}
\end{center}

\newpage

\section{Regras Oficiais Comumente Ignoradas}

Durante uma partida casual de UNO, uma variedade de regras oficiais costuma ser ignorada por diversos motivos: ignorância quanto às regras, confusão entre regras oficiais e alternativas, má interpretação de regras oficiais ou a deliberada exclusão de determinadas regras. Esta seção documenta algumas dessas regras que são frequentemente deixadas de fora, seja para lembrar-se delas com facilidade ou para escolher conscientemente por sua exclusão\footnote{Regras que são ignoradas porque uma regra alternativa entra em vigor serão deixadas de fora deste apêndice: tratar-se-á exclusivamente das regras que não entram em vigor sem uma substituição.}.

\begin{description}
\item[Comprar uma carta e jogar uma que já estava na mão]{Segundo as regras oficiais (ver página \pageref{oficiais}), ``Se [uma carta comprada] puder ser jogada, a carta pode ser descartada na mesma vez, mas o jogador não pode jogar uma carta que estiver na sua mão depois de ter feito a compra''. Geralmente os jogadores compram uma carta como forma de estratégia (ver página \pageref{consumismo}), para obter uma carta melhor que as que já têm, mesmo quando alguma destas pode ser jogada. Depois que a compra se revelou infrutífera, ainda assim jogam uma das cartas que poderiam ter jogado antes.}
\item[Determinação da cor no começo do jogo]{Segundo as regras oficiais, ``Se um Curinga for aberto no começo do jogo, o jogador à esquerda do distribuidor determina a cor que dará sequência ao jogo''. Em geral, a pessoa que determina a cor costuma depender do modo como as cartas são distribuídas, e a regra é ignorada em favor de alguma preferência do grupo.}
\item[Sem duvidar]{Uma das minúcias do UNO é que um jogador que jogar uma carta Curinga Comprar Quatro Cartas não pode fazê-lo se tiver em mãos uma carta de \emph{mesma cor} que a pilha de descarte. Segundo as regras oficiais, ``um jogador que possuir um Curinga Comprar Quatro Cartas pode tentar blefar e jogar uma carta inapropriada, mas certas regras se aplicam se ele for apanhado''. Muitas vezes pode-se esquecer desta propriedade do jogo, e nenhum jogador duvidar da carta jogada.}
\item[Comprar depois do fim]{Segundo as regras oficiais, ``se a última carta jogada em uma partida for a carta Comprar Duas Cartas ou o Curinga Comprar Quatro Cartas, o próximo jogador precisa comprar 2 ou 4 cartas, respectivamente'', e essas cartas são contadas ``na totalização dos pontos''. Por vezes os jogadores pensam que após o fim da partida mais nenhuma penalidade se aplica.}
\item[Palpites]{Segundo as regras oficiais, ``jogadores que derem palpites aos outros jogadores precisam comprar 2 cartas do MONTE''. Esta regra é comumente ignorada, pois os jogadores costumam colaborar entre si com informação --- especialmente quando um dos jogadores está com apenas uma carta na mão.}
\end{description}

\newpage

\section{Ensinando UNO de forma rápida}

Aqui está um guia prático sobre como ensinar UNO para leigos de forma rápida:

\begin{enumerate}
\item{Explique que cada jogador começa com 7 cartas e vai descartando até não sobrar nenhuma na mão, e que quando a pessoa fica com uma carta na mão, deve dizer ``UNO''. Se alguém vê que ela não o fez e a acusa, ela deve comprar duas cartas. Quando o jogo acaba, os pontos das mãos dos adversários são contados e vão para o vencedor: o jogo acaba quando alguém chega aos 500 pontos.}
\item{Explique que você pode descartar uma carta de mesmo número, mesmo símbolo ou mesma cor que a da pilha de descarte. Se não houver nada a ser jogado, você compra uma carta; se puder jogá-la, você pode jogá-la. Se não, o próximo jogador inicia a vez.}
\item{Explique a função das Cartas de Ação e dos Curingas (e sua pontuação equivalente no término de uma partida).}
\item{Diga que existem punições caso alguém jogue um Curinga Comprar Quatro tendo uma carta de mesma cor para jogar, caso alguém dê palpites ou caso jogue equivocadamente.}
\item{Explique eventuais regras alternativas e clarifique que nem todos jogarão com estas regras. É uma boa ideia também jogar primeiro o jogo oficial e depois introduzir regras alternativas.}
\end{enumerate}

Explicar isso fazendo um jogo-exemplo entre pessoas que já sabem jogar e ir explicando os mecanismos do jogo enquanto o aprendiz observa é também uma excelente ideia.

\newpage

\section{Paciência UNO}

Os jogos de Paciência são famosos; são jogos de cartas em que apenas uma pessoa joga, e há uma maneira de usar o baralho de UNO para jogar uma espécie de jogo no estilo de Paciência --- uma mistura dos conhecidos jogos Klondike e Aranha, que ficaram famosos a partir da popularização de computadores pessoais.

Em primeiro lugar, é preciso tirar todos os Curingas e todas as cartas 0 do Baralho. Depois, você deve ver a carta Comprar Duas como o Valete (o \emph{J} do baralho comum), o Inverter como a Dama ou Rainha (o \emph{Q} do baralho comum), e o Pular como o Rei (o \emph{K} do baralho comum). Para se lembrar disso, pense que que Comprar Duas (C), Inverter (I) e Pular (P) estão, nessa ordem, em ordem alfabética por suas primeiras letras, e estão representando as cartas do baralho normal em ordem crescente.

Depois, é preciso entender também que não há carta 10, e portanto logo após o 9 vem o Comprar Duas.

Depois, distribua as cartas na mesa em uma linha horizontal da seguinte maneira: uma carta virada para cima na primeira coluna, uma virada para baixo e outra virada para cima na segunda coluna, duas viradas para baixo e uma virada para cima na terceira coluna, e assim sucessivamente, até que a última coluna tenha seis cartas viradas para baixo e uma carta virada para cima (o número de cartas viradas para baixo começa com 0 e aumenta 1 a cada coluna até o fim; há sempre apenas uma carta virada para cima em cada coluna). A primeira coluna fica à extrema esquerda; a última, à extrema direita.

As cartas que sobrarem são o Monte. Se você consegue ver alguma carta 1, você pode colocá-la em um dos 8 espaços acima das colunas que devem ficar reservados para as sequências de cartas. Assim que você tira uma carta visível de cima de sua pilha de cartas, você pode revelar a primeira das cartas viradas para baixo.

O objetivo é criar oito sequências de cartas de mesma cor (indo do 1 até o Pular) e empilhá-las acima das colunas. Primeiro você deve colocar lá a carta 1, e assim que o fizer, pode colocar a 2, e depois a 3, e assim por diante, até concluir uma sequência.

Para ir fazendo isso, você pode mover uma carta nas colunas para outra nas colunas, se elas forem subsequentes e de mesma cor: ou seja, um 7 azul pode ser colocado embaixo de um oito azul, e estas duas cartas podem, conjuntamente, serem colocadas debaixo de um nove azul. Com essa movimentação você vai oganizando as cartas e descobrindo as cartas que estão viradas para baixo.

Quando você não tiver mais o que mover nas colunas (ou mesmo se o tiver, é claro), você vira uma carta do Monte. Se ela puder ser colocada nos espaços acima das colunas ou debaixo de alguma carta das colunas, você pode fazê-lo. Se não, vire mais uma carta do Monte. Quando o Monte acabar, você pode virá-lo todo para baixo novamente e recomeçar a virar suas cartas. Isso pode ser feito quantas vezes forem necessárias.

Assim que uma coluna ficar vazia, apenas uma carta Pular pode preencher seu lugar. O jogo termina quando você consegue organizar oito sequências de mesma cor.